\documentclass[11pt]{article}

% Paquetes
%===================================================================================================

% Establecemos los márgenes
\usepackage[a4paper, margin=1in]{geometry}

% Separacion entre parrafos
\setlength{\parskip}{1em}

% Paquete para incluir codigo
\usepackage{listings}

% Paquete para incluir imagenes
\usepackage{graphicx}
\graphicspath{ {./Imagenes/} }

% Para fijar las imagenes en la posicion deseada
\usepackage{float}

% Para que el codigo acepte caracteres en utf8
\lstset{literate=
  {á}{{\'a}}1 {é}{{\'e}}1 {í}{{\'i}}1 {ó}{{\'o}}1 {ú}{{\'u}}1
  {Á}{{\'A}}1 {É}{{\'E}}1 {Í}{{\'I}}1 {Ó}{{\'O}}1 {Ú}{{\'U}}1
  {à}{{\`a}}1 {è}{{\`e}}1 {ì}{{\`i}}1 {ò}{{\`o}}1 {ù}{{\`u}}1
  {À}{{\`A}}1 {È}{{\'E}}1 {Ì}{{\`I}}1 {Ò}{{\`O}}1 {Ù}{{\`U}}1
  {ä}{{\"a}}1 {ë}{{\"e}}1 {ï}{{\"i}}1 {ö}{{\"o}}1 {ü}{{\"u}}1
  {Ä}{{\"A}}1 {Ë}{{\"E}}1 {Ï}{{\"I}}1 {Ö}{{\"O}}1 {Ü}{{\"U}}1
  {â}{{\^a}}1 {ê}{{\^e}}1 {î}{{\^i}}1 {ô}{{\^o}}1 {û}{{\^u}}1
  {Â}{{\^A}}1 {Ê}{{\^E}}1 {Î}{{\^I}}1 {Ô}{{\^O}}1 {Û}{{\^U}}1
  {ã}{{\~a}}1 {ẽ}{{\~e}}1 {ĩ}{{\~i}}1 {õ}{{\~o}}1 {ũ}{{\~u}}1
  {Ã}{{\~A}}1 {Ẽ}{{\~E}}1 {Ĩ}{{\~I}}1 {Õ}{{\~O}}1 {Ũ}{{\~U}}1
  {œ}{{\oe}}1 {Œ}{{\OE}}1 {æ}{{\ae}}1 {Æ}{{\AE}}1 {ß}{{\ss}}1
  {ű}{{\H{u}}}1 {Ű}{{\H{U}}}1 {ő}{{\H{o}}}1 {Ő}{{\H{O}}}1
  {ç}{{\c c}}1 {Ç}{{\c C}}1 {ø}{{\o}}1 {å}{{\r a}}1 {Å}{{\r A}}1
  {€}{{\euro}}1 {£}{{\pounds}}1 {«}{{\guillemotleft}}1
  {»}{{\guillemotright}}1 {ñ}{{\~n}}1 {Ñ}{{\~N}}1 {¿}{{?`}}1 {¡}{{!`}}1
}

% Para que no se salgan las lineas de codigo
\lstset{breaklines=true}

% Para que los metadatos que escribe latex esten en español
\usepackage[spanish]{babel}

% Para la bibliografia
% Sin esto, los enlaces de la bibliografia dan un error de compilacion
\usepackage{url}

% Para mostrar graficas de dos imagenes, cada una con su caption, y con un caption comun
\usepackage{subcaption}

% Simbolo de los numeros reales
\usepackage{amssymb}

% Para que los codigos tengan una fuente distinta
\usepackage{courier}

\lstdefinestyle{CustomStyle}{
  language=Python,
  numbers=left,
  stepnumber=1,
  numbersep=10pt,
  tabsize=4,
  showspaces=false,
  showstringspaces=false
  basicstyle=\tiny\ttfamily,
}

% Para incluir tablas en csv
\usepackage{csvsimple}

% Para referenciar secciones usando el nombre de las secciones
\usepackage{nameref}

% Para enumerados dentro de enumerados
\usepackage{enumitem}

% Para mejores tablas
\usepackage{tabularx}

% Para poder tener el mismo identificador en dos tablas separadas
\usepackage{caption}

% Mostrar la página de las referencias en el indice del documento
\usepackage[nottoc,numbib]{tocbibind}

% Para mostrar las matrices
\usepackage{amsmath}

% Metadatos del documento
%===================================================================================================
\title{
    {Aprendizaje Automático - Proyecto Final}\\
    {\emph{Facebook comment volume prediction}}
}

\author{
    {Sergio Quijano Rey - 72103503k}\\
    {sergioquijano@correo.ugr.es} \\
    {} \\
    {Lucía Salamanca López - 77185623s} \\
    {luciasalamanca@correo.ugr.es}\\
    {} \\
    {4º Doble Grado Ingeniería Informática y Matemáticas}
}

\date{\today}

% Separacion entre parrafos
\setlength{\parskip}{1em}

% Contenido del documento
%===================================================================================================
\begin{document}

% Portada del documento
\maketitle
\pagebreak

% Indice de contenidos
\tableofcontents

% Lista de figuras
\listoffigures

% Lista de tablas
\listoftables

\pagebreak

\section{Identificación del problema}

El problema que hemos escogido para esta práctica final consiste en predecir el número de comentarios que un post en \emph{Facebook} va a recibir horas después de ser publicado. La fuente original de los datos se encuentra en \cite{uci:online}. 

En dicha página, se encuentra también la referencia al \emph{paper} original \cite{original_paper:paper}. En dicho paper se comenta el proceso de extracción de los datos, que generan los \emph{datasets} con los que hemos trabajado.

\subsection{Problema a resolver}

Estamos ante un problema de regresión, pues nuestro objetivo es predecir el número de comentarios (variable en principio no finita) a partir de unos datos de entrada. Por tanto, tenemos una función objetivo que viene dada por:

$$f: \mathbb{X} \rightarrow \mathbb{Y}$$

donde $\mathbb{X}$ es el conjunto de las 53 variables aleatorias de entrada, e $\mathbb{Y}$ es la variable aleatoria de salida, entera.

Por lo tanto, nuestro objetivo es encontrar tanto una clase de funciones $\mathcal{H}$ como un algorito de aprendizaje $\mathcal{A}$ que nos devuelva un $h \in \mathcal{H}$ que se asemeje a la función objetivo, es decir

$$h(x) \approx f(x), \forall x \in \mathbb{X}$$

La noción de que ambas funciones sean aproximadas se especificará en \emph{\ref{eleccion_modelos}. \nameref{eleccion_modelos}}.

\subsection{Descripción de las características}

Para obtener las características el autor del paper \cite{original_paper:paper} ha empleado un \emph{crawler} programado en \lstinline{JAVA} y el lenguaje de consulta de \emph{Facebook} (\lstinline{FQL}), con el que se ha recogido información de las páginas de \lstinline{Facebook} de más interés \cite{original_paper:paper}. A partir de dicha informacióm, solo se consideran los comentarios que fueron publicados en los últimos tres días respecto al momento en el que el \emph{crawler} recolecta los datos. Esto pues se presupone que los posts más antiguos no van a recibir más atención. También se eliminan los posts de los que faltan comentarios o algún detalle necesario, de ahí que en nuestro dataset no tengamos \emph{missing values}, como se mostrará en \emph{\ref{exploracion_datos}. \nameref{exploracion_datos}}.

Como se ha dicho previamente tenemmos 53 características que podemos dividirlas en distintos subgrupos.

\subsubsection{Características de la página}

Las extracción de características gira entorno a una de las características principales de \emph{Facebook}, las páginas de \emph{Facebook}. Por tanto, se extraerán características referentes a distintas páginas de \emph{Facebook} con la intención de predecir el número de comentarios que obtendrá un nuevo \emph{post} en dichas páginas.

Existen cuatro características referentes a las páginas: 

\begin{itemize}
  \item \textbf{Likes de la página}: describe el apoyo de los usuarios hacia ciertos elementos de dicha página, como comentarios, imágenes, estados, posts\ldots

% TODO -- Esto mejorar 
  \item \textbf{Categoría de la página}: define el tipo de entidad o persona sobre la que trata la información tratada en dicha página. El dataset contiene un archivo donde se definen todos los tipos posibles de categorías, como por ejemplo: comercio local, marca comercial, producto, artista, entretenimiento, \ldots

  \item \textbf{\emph{Checkin} de la página}: comprobaciones de actos de presencia en un determinado lugar (sólo es válido para páginas institucionales)

  \item \textbf{\emph{Talking About} de la página}: número de usuarios que vuelven a la página tras darle a like a dicha página. Por volver a la página se entienden actividades como comentar, dar like a un post, compartir un post\ldots
\end{itemize}

\subsubsection{Características esenciales}
 
En este subgrupo se incluye el comportamiento de los comentarios en el post en distintos intervalos de tiempo respecto a distintas métricas. Se dividen en cinco secciones que dependen de una referencia temporal fijada. El autor fija esta referencia en 72h después de la publicación del \emph{post}.

\begin{itemize}
  \item \textbf{C1}: número total de comentarios durante las 72 horas previas a la referencia

  \item \textbf{C2}: número total de comentarios durante las  24 horas previas a la referencia 

  \item \textbf{C3}: número total de comentarios entre las 48 y 24 horas previas a la referencia

  \item \textbf{C4}: número total de comentarios en las 24h posteriores a publicar la referencia

  \item \textbf{C5}: diferencia de número total de comentarios entre C2 y C3.

\end{itemize}

A partir de estas características base, calculamos las siguientes estadísticas respecto a los post de una misma página:  mínimo, máximo, media, mediana y desviación típica. Con ello se obtienen 25 características adicionales, que, junto con las cinco características base, conforman 30 características de este subgrupo. 

\subsubsection{Características relativas al día de la semana}

Para representar el día de la semana en el que el post fue publicado y el día con respecto a la referencia temporal se usan indicadores binarios. Existen 14 categorías de este tipo.

Se usan 14 características binarias. 7 de ellas para indicar el día de la semana en la que se publicó el post (todas cero salvo el día de publicación). Las otras 7 características indican el día de la semana de la referencia fijada (de nuevo, 72h después de la publicación del post).

\subsubsection{Otras características básicas}

Incluye 5 características tipo medatadatos del post, como puede ser la longitud del post o el número de veces que se ha compartido. 

\subsection{Exploración del \emph{Dataset}} \label{exploracion_datos}

Con la función \lstinline{explore_dataset}, mostramos algunas estadísticas de las variables aleatorias que componen las columnas de nuestro conjunto de datos. Notar que en esta exploración ya hemos separado el conjunto de \emph{training} y \emph{testing}, con la función \lstinline{split_train_test} que más adelante, en \emph{\ref{preprocesado}. \nameref{preprocesado}}. 

La descripción de las variables se muestra en la siguiente tabla:

\begin{table}[H]
  \centering
  \resizebox{0.8\columnwidth}{!}{%
  \begin{tabular}{|c|c|c|c|c|c|c|c|c|}
    \hline
    \textbf{col}  &         \textbf{mean}&         \textbf{median}&           \textbf{std}&     \textbf{min}&           \textbf{max}&           \textbf{p25}&           \textbf{p75} \\
    \hline
    0    & 1.32e+6&  292911.00&  7.401e+6&    36.0&  4.86e+8&  37149.00&  1.20e+6 \\
    1    & 4.63e+3&       0.00&  2.045e+4&     0.0&  1.86e+5&      0.00&  9.90e+1 \\
    2    & 4.54e+4&    7237.00&  1.237e+5&     0.0&  6.78e+6&    698.00&  5.14e+4 \\
    3    & 2.42e+1&      18.00&  2.001e+1&     1.0&  1.06e+2&      9.00&  3.20e+1 \\
    4    & 1.46e+0&       0.00&  1.872e+1&     0.0&  2.34e+3&      0.00&  0.00e+0 \\
    5    & 4.42e+2&     235.00&  4.958e+2&     0.0&  2.77e+3&     45.00&  7.17e+2 \\
    6    & 5.53e+1&      23.37&  8.565e+1&     0.0&  2.34e+3&      5.51&  7.18e+1 \\
    7    & 3.53e+1&      12.00&  6.842e+1&     0.0&  2.34e+3&      2.00&  4.20e+1 \\
    8    & 6.72e+1&      35.06&  8.117e+1&     0.0&  1.00e+3&      7.88&  1.02e+2 \\
    9    & 1.62e-1&       0.00&  3.337e+0&     0.0&  3.81e+2&      0.00&  0.00e+0 \\
    10   & 2.84e+2&     118.00&  3.758e+2&     0.0&  2.77e+3&     26.00&  4.01e+2 \\
    11   & 2.21e+1&       8.43&  3.599e+1&     0.0&  9.99e+2&      1.91&  2.90e+1 \\
    12   & 7.42e+0&       2.00&  1.980e+1&     0.0&  7.97e+2&      0.00&  8.00e+0 \\
    13   & 4.04e+1&      17.38&  5.450e+1&     0.0&  8.70e+2&      4.10&  6.07e+1 \\
    14   & 2.90e-2&       0.00&  2.209e+0&     0.0&  3.24e+2&      0.00&  0.00e+0 \\
    15   & 2.67e+2&     116.00&  3.257e+2&     0.0&  1.87e+3&     26.00&  3.81e+2 \\
    16   & 1.95e+1&       8.58&  3.062e+1&     0.0&  4.37e+2&      2.03&  2.48e+1 \\
    17   & 4.87e+0&       1.00&  1.311e+1&     0.0&  4.33e+2&      0.00&  5.00e+0 \\
    18   & 3.85e+1&      18.63&  5.041e+1&     0.0&  5.80e+2&      4.09&  5.38e+1 \\
    19   & 1.38e+0&       0.00&  1.642e+1&     0.0&  1.89e+3&      0.00&  0.00e+0 \\
    20   & 4.14e+2&     224.00&  4.719e+2&     0.0&  2.77e+3&     41.00&  6.70e+2 \\
    21   & 5.23e+1&      21.85&  8.017e+1&     0.0&  1.89e+3&      5.21&  6.83e+1 \\
    22   & 3.37e+1&      12.00&  6.459e+1&     0.0&  1.89e+3&      2.00&  4.00e+1 \\
    23   & 6.29e+1&      32.36&  7.611e+1&     0.0&  8.52e+2&      7.60&  9.62e+1 \\
    24   &-2.19e+2&     -92.00&  2.801e+2& -1677.0&  3.81e+2&   -310.00& -2.10e+1 \\
    25   & 2.75e+2&     109.00&  3.730e+2&  -204.0&  2.77e+3&     23.00&  3.79e+2 \\
    26   & 2.58e+0&       0.27&  1.563e+1&  -210.5&  6.39e+2&     -0.48&  2.97e+0 \\
    27   &-1.99e+0&       0.00&  1.237e+1&  -288.0&  7.97e+2&     -2.00&  0.00e+0 \\
    28   & 5.56e+1&      25.54&  7.372e+1&     0.0&  1.33e+3&      5.99&  8.12e+1 \\
    29   & 5.50e+1&      11.00&  1.345e+2&     0.0&  2.34e+3&      2.00&  4.60e+1 \\
    30   & 2.20e+1&       2.00&  7.666e+1&     0.0&  2.09e+3&      0.00&  1.20e+1 \\
    31   & 1.94e+1&       0.00&  6.916e+1&     0.0&  1.59e+3&      0.00&  9.00e+0 \\
    32   & 5.20e+1&      10.00&  1.258e+2&     0.0&  2.18e+3&      2.00&  4.40e+1 \\
    33   & 2.64e+0&       0.00&  9.292e+1& -1277.0&  2.09e+3&     -6.00&  3.00e+0 \\
    34   & 3.51e+1&      35.00&  2.099e+1&     0.0&  7.20e+1&     17.00&  5.30e+1 \\
    35   & 1.64e+2&      97.00&  3.826e+2&     0.0&  2.14e+4&     38.00&  1.72e+2 \\
    36   & 1.17e+2&      13.00&  1.023e+3&     1.0&  1.44e+5&      2.00&  6.00e+1 \\
    37   & 0.00e+0&       0.00&  0.000e+0&     0.0&  0.00e+0&      0.00&  0.00e+0 \\
    38   & 2.37e+1&      24.00&  2.008e+0&     1.0&  2.40e+1&     24.00&  2.40e+1 \\
    39   & 1.22e-1&       0.00&  3.273e-1&     0.0&  1.00e+0&      0.00&  0.00e+0 \\
    40   & 1.43e-1&       0.00&  3.506e-1&     0.0&  1.00e+0&      0.00&  0.00e+0 \\
    41   & 1.49e-1&       0.00&  3.566e-1&     0.0&  1.00e+0&      0.00&  0.00e+0 \\
    42   & 1.56e-1&       0.00&  3.622e-1&     0.0&  1.00e+0&      0.00&  0.00e+0 \\
    43   & 1.45e-1&       0.00&  3.525e-1&     0.0&  1.00e+0&      0.00&  0.00e+0 \\
    44   & 1.45e-1&       0.00&  3.523e-1&     0.0&  1.00e+0&      0.00&  0.00e+0 \\
    45   & 1.37e-1&       0.00&  3.443e-1&     0.0&  1.00e+0&      0.00&  0.00e+0 \\
    46   & 1.43e-1&       0.00&  3.502e-1&     0.0&  1.00e+0&      0.00&  0.00e+0 \\
    47   & 1.31e-1&       0.00&  3.382e-1&     0.0&  1.00e+0&      0.00&  0.00e+0 \\
    48   & 1.37e-1&       0.00&  3.443e-1&     0.0&  1.00e+0&      0.00&  0.00e+0 \\
    49   & 1.50e-1&       0.00&  3.572e-1&     0.0&  1.00e+0&      0.00&  0.00e+0 \\
    50   & 1.49e-1&       0.00&  3.564e-1&     0.0&  1.00e+0&      0.00&  0.00e+0 \\
    51   & 1.44e-1&       0.00&  3.518e-1&     0.0&  1.00e+0&      0.00&  0.00e+0 \\
    52   & 1.44e-1&       0.00&  3.514e-1&     0.0&  1.00e+0&      0.00&  0.00e+0 \\
    53   & 7.43e+0&       0.00&  3.588e+1&     0.0&  1.30e+3&      0.00&  3.00e+0 \\
    \hline
  \end{tabular}
}
\caption{Exploración estadística de los atributos del conjunto de entrenamiento}
\end{table}

En la tabla no hemos mostrado la estadística de \emph{missing values}, pues como ya hemos comentado, no tenemos datos faltantes en este \emph{dataset}. Tampoco mostramos la columna en la que se muestra el tipo, pues todas las variables son o bien flotantes o bien enteras, y por tanto no es necesaria una técnica de codificación de variables categóricas como puede ser \emph{one hot encoding}.

En esta tabla queda clara la necesitada de realizar un proceso de estandarización o bien de normalización. Los modelos que vamos a emplear son muy sensibles a variables en escalas diferentes. Otros directamente se comportam mucho mejor cuando tenemos media cero y desviación típica uno. Estas escalas diferentes se pueden ver, por ejemplo, con la columna 0 que se mueve en el rango $[36.0, 4.86 * 10^8]$, mientras que la columna 43 se mueve en el rango $[0, 1]$

\pagebreak

\section{Preprocesamiento de los datos} \label{preprocesado}

\pagebreak

\section{Selección del modelo} \label{eleccion_modelos}

\pagebreak

\
% Bibliografia
\bibliography{./References}
\bibliographystyle{ieeetr}

\end{document}
